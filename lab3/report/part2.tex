\section{Теория}
\subsection{Представление данных}
\begin{flushleft}
	В первую очередь представим данные таким образом, чтобы применить понятия статистики данных с интревальной неопределенностью. Один из распространненых способов получения интервальных результатов в первичных измерениях $-$ это "обинтерваливание" точечных значений, когда к точечному базовому значению $\dot{x}$, которое считывается по показаниям измерительного прибора прибавляется интервал погрешности $\epsilon$:
	\begin{equation}
		x \,=\, \dot{x}\,+\,\epsilon
		\label{3}
	\end{equation}
	Интервал погрешности зададим как
	\begin{equation}
		\epsilon\,=\,[-\xi,\,\xi].
		\label{4}
	\end{equation}
	В конкретных измерениях примем $\xi\,=\,10^{-4}$ мВ.\\
	Согласно терминологии интервального анализа, рассматриваемая выборка $-$ это вектор интервалов, или интервальный вектор $x \,=\, (x_1,x_2,x_3,...)$.\\
	Информационным множеством в случае оценивания единичной физической величины по выборке итервальных данных будет также интервал, который называют также информационным интервалом. Неформально говоря, это интервал, содержащий значения оцениваемой величины, которые "совместны" с измерениями выборки.
\end{flushleft}

\subsection{Простая линейная регрессия}
\begin{flushleft}
	Регрессионную модель описания данных называют простой линейной, если заданный набор данных аппроксимируется прямой с внесенной добавкой в виде некоторой нормально распределенной ошибки:
	\begin{equation}
		y_i \,=\,\beta_{0}\,+\,\beta_{1}*x_i \,+\,\epsilon_{i}, \, i\in \bar{1,n}
		\label{5}
	\end{equation}
	где $\beta_{0},\beta_{1}\,-$ параметры подлежащие оцениванию. В данном случае рассматриваем модель без внесения добавки.
\end{flushleft}

\subsection{Метод наименьших модулей}
\begin{flushleft}
	Данный метод основан на минимизации $l^1$-нормы разности последовательностей полученных экспериментальных данных ${y_n}$ и значений аппроксимирующей функции $f({x_n})$:
	\begin{equation}
		||f({x_n})-{y_n}||_{l^1}\,\rightarrow\, min
		\label{6}
	\end{equation}
	В данном случае мы ставим задачу линейного программирования таким образом, чтобы найти не только $\beta_{0}$ и $\beta_{1}$, но и вектор $w$ множителей коррекции. Тогда задача ставится в следующем виде:
	\begin{equation}
		\sum|w_i|\,\rightarrow\, min, \, i \in \bar{1,n}
		\label{7}
	\end{equation}
	При ограничениях:
	\begin{equation}
		\beta_{0}\,+\,\beta_{1}*x_i \,-\, w_i*\xi \leq\, y_i, \, i \in \bar{1,n}
		\label{8}
	\end{equation}
	\begin{equation}
		\beta_{0}\,+\,\beta_{1}*x_i \,+\, w_i*\xi \leq\, y_i, \, i \in \bar{1,n}
		\label{9}
	\end{equation}
\end{flushleft}

\subsection{Предварительная обработка данных}
\begin{flushleft}
	Из графического представления выборок ясно, что для оценки коэффициента калибровки необходима предварительная обработка данных. Для этого зададимся линейной моделью дрейфа:
	\begin{equation}
		Lin_{1,2} \,=\, A_{1,2}\,+\, B_{1,2}*n \,+\,\epsilon_{i}, \, n\in \bar{1,N}.
		\label{10}
	\end{equation}
	Поставим задачу линейного программирования \ref{7}-\ref{9} и найдем коэффициенты $A_{1,2},\, B_{1,2}$ и вектор $w_{1,2}$ множителей коррекции данных. Множитель коррекции необходим для того чтобы получить данные, согласующиеся с полученной линейной моделью дрейфа:
	\begin{equation}
 		I_{1,2}^f(n) \,=\, \dot{x}(n) \,+\,\epsilon*w_{1,2}(n), \, n\in \bar{1,N}.
	 	\label{11}
	\end{equation}
 	После построения линейной модели дрейфа необходимо построить "спрямленные" данные выборки:
 	\begin{equation}
 		I_{1,2}^{с}(n) \,=\, I_{1,2}^f(n) \,-\, B_{1,2}*n, \, n\in \bar{1,N}.
 		\label{12}
 	\end{equation}
\end{flushleft}

\subsection{Коэффициент Жаккара}
\begin{flushleft}
	Нами рассматривается модификация индекса Жаккара для интервальных данных:
	\begin{equation}
		JK(x)\,=\,\frac{wid(\cap{x_i})}{wid(\cup{x_i})}
		\label{13}
	\end{equation}
	В качестве меры рассматривается ширина интервала, а вместо пересечения и объединения $-$ взятие минимума и максимума по включению двух величин в интервальной арифметике. Поскольку минимум по включению может быть неправильным интервалом, коэффициент нормирован на промежутке [-1;1].
\end{flushleft}

\subsection{Процедура оптимизации}
\begin{flushleft}
	Для поиска оптимального параметра калибровки поставим задачу максимизации:
	\begin{equation}
		JK(x_{1\oplus2})\,\rightarrow\, max
		\label{14}
	\end{equation}
	где $x_{1\oplus2}$ выборка полученная как конкатенация двух выборок
	\begin{equation}
		x_{1\oplus2}\,=\, I_1^f*R \oplus I_2^f.
		\label{15}
	\end{equation}
	При этом, поскольку знак коэффициента Жаккара может свидетельствовать о совместности двух выборок (исходя из правильности минимума по включению), в качестве интервала для $R_{21}$ можно рассматривать область где $JK(R)\geq0$.
\end{flushleft}